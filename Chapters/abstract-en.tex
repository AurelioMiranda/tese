%!TEX root = ../template.tex
%%%%%%%%%%%%%%%%%%%%%%%%%%%%%%%%%%%%%%%%%%%%%%%%%%%%%%%%%%%%%%%%%%%%
%% abstract-en.tex
%% NOVA thesis document file
%%
%% Abstract in English([^%]*)
%%%%%%%%%%%%%%%%%%%%%%%%%%%%%%%%%%%%%%%%%%%%%%%%%%%%%%%%%%%%%%%%%%%%

\typeout{NT FILE abstract-en.tex}%

The automatic generation of code from templates is a widely adopted approach in the industry to reduce costs and increase software reliability. However, this generation has to be highly configurable to meet specific requirements, such as project coding practices, compatibility with APIs and performance optimizations.

RAMSES is an Architecture Analysis and Design Language (AADL) code generation tool that fully automates the process of converting AADL models into code to support the design of embedded and cyber-physical systems. The most significant advantage of RAMSES is its ability to automatically generate code from high-level models, eliminating implementation details and providing better portability and reusability. However, as industrial systems become increasingly diverse, the need to adapt to specific industrial environments requires an adaptable configuration of the generated code.

This project proposes the design and implementation of a configuration language for RAMSES, enabling code generation to be customized according to specific requirements of each industry. The steps involve defining the syntax and semantics of the language, integrating it into RAMSES and testing it through industrial scenarios.

Throughout the project, various scenarios will be considered to demonstrate the effectiveness of the solution in comparison with other tools in this context. The aim is to provide an intuitive and helpful tool that can be used to make the language adaptable.

% Palavras-chave do resumo em Inglês
% \begin{keywords}
% Keyword 1, Keyword 2, Keyword 3, Keyword 4, Keyword 5, Keyword 6, Keyword 7, Keyword 8, Keyword 9
% \end{keywords}
\keywords{
  Code Generation \and
  AADL \and
  RAMSES \and
  Industrial Automation
}
