%!TEX root = ../template.tex
%%%%%%%%%%%%%%%%%%%%%%%%%%%%%%%%%%%%%%%%%%%%%%%%%%%%%%%%%%%%%%%%%%%%
%% abstract-fr.tex
%% NOVA thesis document file
%%
%% Abstract in French
%%%%%%%%%%%%%%%%%%%%%%%%%%%%%%%%%%%%%%%%%%%%%%%%%%%%%%%%%%%%%%%%%%%%

\typeout{NT FILE abstract-fr.tex}%

La génération automatique de code à partir de modèles est une approche largement adoptée dans l'industrie pour réduire les coûts et augmenter la fiabilité des logiciels. Cependant, cette génération doit être hautement configurable pour répondre à des exigences spécifiques, telles que les bonnes pratiques de codage, la compatibilité avec les API existantes et l'optimisation des performances.

RAMSES est un outil de génération de code AADL qui automatise entièrement le processus de conversion des modèles AADL en code pour soutenir la conception de systèmes embarqués et cyber-physiques. L'avantage le plus significatif de RAMSES est sa capacité à générer automatiquement du code à partir de modèles de haut niveau, en éliminant les détails d'implémentation et en offrant une meilleure portabilité et réutilisation. Cependant, les systèmes industriels devenant de plus en plus diversifiés, la nécessité de s'adapter à des environnements industriels spécifiques requiert une configuration adaptable du code généré.

Ce projet propose la conception et la mise en œuvre d'un langage de configuration pour RAMSES afin de permettre la personnalisation de la génération de code en fonction des exigences de chaque industrie. Les étapes consistent à définir la syntaxe et la sémantique du langage, à intégrer le langage dans RAMSES et à le tester à travers des scénarios industriels.

Tout au long du projet, divers scénarios seront envisagés pour démontrer l'efficacité de la solution par rapport à d'autres outils dans ce contexte. L'objectif est de produire un outil intuitif et utile qui peut être utilisé pour rendre le langage adaptable.

% Palavras-chave do resumo em Francês
\begin{keywords}
Génération de code \and
AADL \and
RAMSES \and
Automatisation industrielle
\end{keywords}
% to add an extra black line
