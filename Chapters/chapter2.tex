%!TEX root = ../template.tex
%%%%%%%%%%%%%%%%%%%%%%%%%%%%%%%%%%%%%%%%%%%%%%%%%%%%%%%%%%%%%%%%%%%%
%% chapter2.tex
%% NOVA thesis document file
%%
%% Chapter with the template manual
%%%%%%%%%%%%%%%%%%%%%%%%%%%%%%%%%%%%%%%%%%%%%%%%%%%%%%%%%%%%%%%%%%%%

\typeout{NT FILE chapter2.tex}%


\chapter{Background and Related Work} \label{sec:background_and_relatedwork}


%change this maybe
\epigraph{ \textit{“The more you automate, the more you can focus on design, not on mechanics.”}\ — Some Systems Engineer }


\section{Model-Based Engineering (MBE) and AADL} \label{sec:mbe_and_aadl}

Model-Based Engineering (MBE) has become a central methodology for the design of complex embedded systems. By putting high-level abstractions at the center, MBE enables engineers to manage system complexity through formal models rather than low-level code from the start. This abstraction is particularly critical in embedded systems, where hardware constraints and timing requirements must be closely integrated with software behavior.
\par
In the context of embedded systems, MBE facilitates early validation of design decisions, much earlier than hardware exists or code is written. Engineers can model interactions, analyze performance bottlenecks, and verify compliance with safety and reliability standards — all at the model level.
\par
One of the most important participants in this strategy is the \textbf{Architecture Analysis \& Design Language (AADL)}. AADL is a formal hardware/software co-design modeling language. It gives precise semantics to model the architecture and behavior of embedded systems, ranging from processor bindings and memory layouts to communication buses and task scheduling.
\par
AADL is not only strong in its description power but stronger in being capable of supporting early analysis of non-functional properties such as timing, reliability, and safety constraints. This is very well suited to industries such as aerospace, automotive, and defense, where such considerations are a given.
\par

\begin{tcolorbox}[colback=blue!5, colframe=blue!40!black] With AADL adoption, developers are able to early validate system architectures, preventing downstream integration risks and costly late-stage design modifications.  \end{tcolorbox}

\par
% close paragraph with "foreshadowing"
In this thesis, AADL is utilized as the base modeling language. Its formality and tool support, particularly within RAMSES, will facilitate automatic translation of abstract designs to the execution code and bridging of the system design and implementation gap.


\section{RAMSES: A Code Generator for AADL} \label{sec:ramses}

RAMSES (\textit{Reusable AADL Model Simulation Execution Support}) is an M2T transformation tool with a focus on code generation from AADL models. Part of the greater Eclipse ecosystem, RAMSES automates the transformation of architectural models into deployable source code, effectively achieving the MBE dream of model-driven automation.
\par 
RAMSES now supports code generation in both \textbf{C} and \textbf{C++}. This makes it possible to use it in a broad variety of embedded development settings, depending on whether the target environment needs low-level procedural programming or more structured, object-oriented design paradigms.
\par 
The tool does this by systematically correlating AADL model elements to their corresponding code structures. Processors, threads, communication channels, and data components declared in AADL are mapped to their code counterparts, so much of the boilerplate and scaffolding code otherwise written by hand being done automatically.

% tactical pause w/ fun fact about ramses
\begin{tcolorbox}[colback=green!5, colframe=green!40!black] Automation through RAMSES accelerates development and reduces human error, especially in large-scale embedded projects. \end{tcolorbox}

Yet, despite its advantages, RAMSES is not flawless. Its transformation logic is currently hardcoded, so developers have little control over customizing or fine-tuning the code structure generated without having to alter the tool itself. This rigidity becomes a performance bottleneck in projects that involve customized code structures, strict following of certain coding guidelines, or multi-variant code generation.
\par 
% closing chapter?
This constraint will be a discussed throughout this thesis. In subsequent chapters, we will return to RAMSES to discuss its architecture in greater depth and look at potential ways to make it more configurable. ((Should i discuss architecture here actually? -A))





\section{Code Generators in AADL and Beyond} \label{sub:code_generators}

% intro
While RAMSES plays a central role in the AADL toolset, it is by no means the only one in the world of model-based code generation. There are long-established solutions both inside and outside the AADL universe with their own capabilities and niches.

% \subsection*{Ocarina}

% maybe include a sub about ocarina?


\subsection*{Simulink Code Generation}

Simulink is a flagship Model-Based Design solution, particularly in control systems engineering, developed by MathWorks. In comparison with the tightly integrated AADL-inherent RAMSES, Simulink is backed by a graphic modeling framework of dynamic systems, and the production of code becomes straightforward with software like Simulink Coder and Embedded Coder.

Key aspects of Simulink code generation are:
 \begin{itemize} 
 	\item \textbf{Model-Based Design:} Control systems can be graphically designed, simulated, and validated by engineers before code generation. 
 	\item \textbf{Template-Based Generation:} Code is generated from pre-defined templates to enable integration into existing software platforms. 
 	\item \textbf{Customization and Extensions:} Developers can customize generation patterns and integrate generated code into larger legacy codebases. 
 \end{itemize}

Simulink is especially well-suited for rapid prototyping and tight integration with hardware-in-the-loop testing, and thus it is a favorite among automotive and aerospace industries.


\subsection*{AUTOSAR: Automotive Industry Standard}

AUTOSAR (AUTomotive Open System ARchitecture) is an open, standardized automotive embedded system software architecture. Unlike RAMSES and Simulink, AUTOSAR is more than a code generation tool - it is an entire methodology and collection of specifications.

Highlights of AUTOSAR include: 
\begin{itemize} 
	\item \textbf{Industry Standardization:} Strongly accepted across the automotive industry for interoperability and reliability.
	\item \textbf{High Configurability:} Facilitates extensive parameterization and reusable software components. 
	\item \textbf{Tool Integration:} AUTOSAR-tool-supporting tools mechanize configuration, validation, and code generation tasks. 
\end{itemize}

\begin{tcolorbox}[colback=green!5, colframe=green!40!black] AUTOSAR’s configurability contrasts with RAMSES, which currently relies on hardcoded transformations, limiting its flexibility. \end{tcolorbox}


\subsection*{RAMSES vs. Simulink/TargetLink: A Comparison}

It is important to mention a fundamental distinction: 
\begin{itemize} 
	\item \textbf{RAMSES} automates the model-to-code transformation for AADL but struggles with configurability. 
	\item \textbf{Simulink and TargetLink} (a MathWorks partner product) offer multivariant configuration support, enabling developers to generate optimized code for a variety of targets from the same model.
\end{itemize}

This comparative remark underscores the motivation for this thesis: examining how tools like Acceleo can bridge this configurability gap in RAMSES.

\section{Acceleo and Model-to-Text Transformations} \label{sec:folders_and_files}

To counter the configurability limitations observed in tools like RAMSES, we turn to specialized model-to-text (M2T) transformation technologies. Among these, Acceleo is a highly promising candidate.

\subsection*{Acceleo: An Overview}

Acceleo is an open-source, template-based Eclipse family M2T transformation tool. Its thought model is based on the mapping of formal models (typically in EMF — Eclipse Modeling Framework format) to text artifacts like source code, documentation, or configuration files.

Major benefits of Acceleo are: 
\begin{itemize} 
	\item \textbf{Template-Based Transformation:} Developers specify templates that describe how the elements of a model should be translated into textual form.
	\item \textbf{Strong Eclipse Integration:} Acceleo offers robust integration with the Eclipse IDE, providing instant feedback, syntax coloring, and incremental generation.
	\item \textbf{Structured Code Generation:} Well suited for generating structured, maintainable C/C++ code from high-level models.
\end{itemize}

\begin{tcolorbox}[colback=blue!5, colframe=blue!40!black] Acceleo gives developers the ability to tweak code generation patterns, making the generated codebase more flexible and maintainable. \end{tcolorbox}

\subsection*{Acceleo’s Role in This Thesis}

For this project, Acceleo serves as the basis for enhancing RAMSES' configurability. Through delegating transformation logic to Acceleo templates, we have the aim of: 
\begin{itemize} 
	\item Isolate transformation rules from RAMSES' internal code.
	\item Allow easy extension and modification of code generation patterns.
	\item Facilitate adherence to industrial standards such as MISRA C/C++.
\end{itemize}

This plan promises to transform RAMSES into a more flexible and maintainable toolchain component from one that is rigid code generating.




\section{Existing Work on Configurable Code Generation} \label{sec:configurable_generation}

The search for flexible and customizable code generation is not unique to this thesis. In most domains, tools and techniques have been created to solve the problem of generating high-quality, customizable code from models.

\subsection*{Template-Based Approaches}

Template-based code generation remains the foundation in this field. Some good examples of such tools are \textbf{Acceleo} and \textbf{Simulink templates}: 

\begin{itemize} 
	\item \textbf{Acceleo} allows explicit control of the structure and style of the generated code, making it highly suitable for projects in which compliance with some coding standards or architecture patterns is essential. 
	\item \textbf{Simulink Templates} offers programmers the means to declare patterns of reusable code, with uniform look and feel across several projects and support for custom toolchains and legacy systems.
\end{itemize}

These approaches allow programmers to mold the auto-generated code towards project-specific applications without downgrading underlying models bridging the gap between automated generation and hand-coding, combining efficiency with flexibility.

\subsection*{Hook Functions in TargetLink}

TargetLink, another market leader in code generation tools, comes with the concept of \textbf{hook functions} — pre-compiled points of extension within the generated code that allow developers to plug in their own logic. The facility is most handy in a number of situations. For example, it eases the integration with legacy APIs or platform-dependent libraries and allows developers to add extensions without altering the primary generated code.
\par
In addition, hook functions have the benefit of being customizable without compromising maintainability or upgradability of the generated code. When models evolve, code under it can remain unchanged while introducing custom logic using these extension points. This solution offers a clean trade-off between extending the generated code and offering its long-term maintainability with less effort for future upgrades.

\subsection*{OpenModelica and Multi-Variant Generation}

\textbf{OpenModelica} introduces a higher degree of configurable generation with its support for \textbf{multi-variant code generation}. Through this, engineers are able to:
\begin{itemize} 
	\item Create multiple variants of code based on a common base model.
	\item Tailor outputs for various deployment contexts, hardware configurations, or performance constraints.
\end{itemize}

This variability is completely indispensable in automobile or aircraft production companies, for example, where a single product line might encompass several hardware targets or safety classes.


\subsection*{The Case for Configurability in RAMSES}

Despite its strengths, RAMSES currently has no mechanism for fine-grained extension and configuration.
Specifically:

\begin{itemize} 
	\item Transformation rules are hard-coded, which restricts flexibility.
	\item There is no native support for multi-variant generation or integration points like hook functions. 
\end{itemize}

Including configurability in RAMSES would offer several benefits. It would facilitate the generation of custom code for different deployment environments, making it easier to adapt to specific hardware environments or performance requirements. In addition, the flexibility would simplify maintenance and development of the transformation logic, allowing the tool to better support changing development needs. Finally, by making RAMSES more configurable, it would be easier to interface with industry standards and legacy systems, rendering the tool flexible and applicable in high-speed industries.

\begin{tcolorbox}[colback=green!5, colframe=green!40!black] By adopting template-based generation, RAMSES can evolve into a dynamic, future-proof tool to meet growing embedded system development demands. \end{tcolorbox}

\subsection*{Towards MISRA C/C++ Compliance}

Finally, a central element of code generation in configurable code generation, particularly in the field of safety-critical application domains, is to generate \textbf{standard-compliant code}. Strict requirements for safe, portable, and reliable embedded software are presented by the MISRA (Motor Industry Software Reliability Association) C and C++ standards.
\par
Compliance to MISRA plays several principal roles: it enhances software safety by minimizing the likelihood of undefined behavior and runtime errors, guarantees that development processes meet the high standards demanded by industries such as the automobile and aerospace industries where in some instances compliance is mandatory, and is readily compatible with existing toolchains, as most static analysis tools are tailored to enforce MISRA rules.
\par
As we integrate configurable generation facilities into RAMSES, we shall ensure that code generated is MISRA C/C++ compliant.

\begin{tcolorbox}[colback=blue!5, colframe=blue!40!black] Flexible code generators need to not just conform to project requirements but also apply vital industry standards such as MISRA to guarantee safety and reliability. \end{tcolorbox}



%
% Please note that
% \begin{center}
%   \textbf{\large this package and template are not official for FCT/NOVA}.
% \end{center}



% \printbibliography[heading=subbibliography, segment=\therefsegment, title={\bibname\ for chapter~\thechapter}]
