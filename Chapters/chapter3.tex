%!TEX root = ../template.tex
%%%%%%%%%%%%%%%%%%%%%%%%%%%%%%%%%%%%%%%%%%%%%%%%%%%%%%%%%%%%%%%%%%%%
%% chapter3.tex
%% NOVA thesis document file
%%
%% Chapter with a short latex tutorial and examples
%%%%%%%%%%%%%%%%%%%%%%%%%%%%%%%%%%%%%%%%%%%%%%%%%%%%%%%%%%%%%%%%%%%%




\chapter{Challenges and Requirements for a Configurable Code Generator}
\label{chap:challenges_and_requirements}

\epigraph{
	In this chapter reveals the limitations of RAMSES' inflexible hardcoded transformation logic, describes the industrial requirements for user-defined, standards compliant and maintainable code generation, and provides the incentive for creating a dedicated configuration language to separate policy from logic to allow user-control over naming, structure, and integration requirements from diverse embedded systems contexts.
}

\section{Current Issues with RAMSES Code Generation}
\label{sec:current_issues}

RAMSES is a powerful tool for automatic source code generation from AADL models, yet it faces significant limitations:

\begin{itemize}
	\item \textbf{Hardcoded Transformation Logic:} RAMSES' code generation policies—naming conventions, API selection, and coding style—are strongly embedded in the transformation rules themselves, meaning that, if a user wants to modify things such as the identifier nomenclature (eg. function names using snake\_case), that has to be done directly in the code, instance by instance. This makes any code modification difficult and prone to errors.
	\item \textbf{Lack of User-Defined Customization:} The generated code is not straightforward for developers to alter in order to fit project-specific requirements, like:
	\begin{itemize}
		\item Naming conventions (e.g., camelCase, snake\_case).
		\item Integration with existing company APIs or legacy libraries.
		\item Coding styles or compliance with internal guidelines.
		\item A couple more [tbd]
	\end{itemize}
	This means that if a developer wants to change something in the generated code, the procedure is tedious as the files need to be located, changed and integrated with the rest of the project\footnote{If a function name is altered, every instance of that name must be altered too, including other files that might use that same function name.}.
\end{itemize}

The rigidity that exists in RAMSES creates barriers to adoption in diverse industrial contexts where flexible code generation is critical.

\section{Industrial Requirements}
\label{sec:industrial_requirements}

Embedded systems development in industry imposes stringent requirements on generated code:

\begin{itemize}
	\item \textbf{Coding Best Practices and Standards:} Many companies enforce strict internal coding guidelines or industry standards (such as MISRA C/C++). These rules affect naming conventions, code layout, error handling, and more.
	\item \textbf{Reuse of Legacy APIs and Libraries:} Projects often rely on existing software assets—proprietary libraries, hardware abstraction layers, or middleware. Seamless integration with these components is mandatory.
	\item \textbf{Maintainability and Readability:} Generated code must be readable and maintainable by developers, supporting debugging, auditing, and certification processes.
	\item \textbf{Adaptability to Target Platforms:} Different embedded targets may have varied constraints requiring code generation adjustments (e.g., memory management strategies, concurrency models).
\end{itemize}

Addressing these needs requires a more flexible and user-driven generation process as industrial embedded software development demands code generators that are not “one size fits all” but configurable to reflect evolving project requirements and standards.



\section{Defining a Configuration Language}
\label{sec:config_language_definition}

To overcome the identified challenges and satisfy industrial requirements, a dedicated configuration language for RAMSES code generation is proposed.

Key considerations for this language include:

\begin{itemize}
	\item \textbf{Scope of Configurability:} The language should enable customization of:
	\begin{itemize}
		\item Naming conventions (variable, function, type names).
		\item Code structure and style (indentation(?), braces style(?), comments).
		\item Legacy code integration.
		\item Memory handling options.
	\end{itemize}
	
	\item \textbf{Specification Mechanism:} Configurations should be specified in a declarative, human-readable format that:
	\begin{itemize}
		\item Is easy to learn and use by developers.
		\item Integrates seamlessly with the RAMSES generation workflow.
		\item Allows inheritance or layering of configurations for reuse and extension.
	\end{itemize}
	
	\item \textbf{Extensibility and Maintainability:} The language and its implementation must support:
	\begin{itemize}
		\item Easy addition of new configuration options.
		\item Clear separation between generation logic and configuration.
		\item Support for validation and error checking of configuration files.
	\end{itemize}
\end{itemize}

\begin{tcolorbox}[colback=green!10, colframe=green!50!black, title=Goal of the Configuration Language]
	Provide a flexible and structured way of accommodating RAMSES' code generation mechanism to a broad spectrum of industrial needs without changing its internal transformation logic.
\end{tcolorbox}



