%!TEX root = ../template.tex
%%%%%%%%%%%%%%%%%%%%%%%%%%%%%%%%%%%%%%%%%%%%%%%%%%%%%%%%%%%%%%%%%%%%
%% chapter5.tex
%% NOVA thesis document file
%%
%% Chapter with lots of dummy text
%%%%%%%%%%%%%%%%%%%%%%%%%%%%%%%%%%%%%%%%%%%%%%%%%%%%%%%%%%%%%%%%%%%%

\typeout{NT FILE chapter5.tex}%

\chapter{System Design and Realization}
\label{cha:impl}


\epigraph{ \textit{This chapter describes the technical implementation of the proposed configuration language for RAMSES. It details how the insights from industrial collaboration informed the architectural decisions and outlines the layered development of features supporting flexible, standards-compliant, and modular code generation.}}


\section{Industry Partner Meeting}
\label{sec:industry_meeting}

The first meeting with an industry partner revealed to be very useful with many insights to note, particularly that a few features were highlighted and prioritized in order to maintain a good development speed and an overall better plan. Most of the previously thought of features (Identifiers, Traceability, Report, etc) were endorsed, however, a new feature was suggested by an expert \gls{ROS} developer: the separation of \gls{ROS} specific code and C++ specific code, meaning that code that is specific to \gls{ROS} can be isolated as it is very similar along the various generated files. This will essentially be done with an interface that will implement \gls{ROS} code on C++ only files, which effectively serves as a separation of concerns.

\begin{tcolorbox}[colback=green!5, colframe=green!40!black] This meeting provided extremely valuable insight as the industry partner has been using \gls{ROS} for over seven years, effectively distilling their experience into actionable knowledge. \end{tcolorbox}


\section{Implementation Plan} 
\label{sec:impl_plan}

In order to better understand what features will be implemented, when and why, the implementation will be done by layers, each layers will depend on things done previously on other layers, hence some features are developed before others.

\bgroup
\rowcolors{1}{}{GhostWhite}
\begin{xltabular}{\textwidth}{X X X}
	\caption{Feature dependency table}
	\label{tab:impl_plan}\\
	\toprule
	\rowcolor{Gainsboro}%
	Feature & Depends On & Notes \\
	\midrule
	Identifiers & – & Core to code structure \\
	Comments & Identifiers & Cheap to implement early \\
	Traceability & Identifiers, Comments & Hard to retrofit later \\
	Report & Traceability & Uses trace data \\
	Standard\par Compliance & Report, Traceability & Enforce compliance early \\
	Dead Code\par Elimination & Traceability, Report & Needs stable generation logic \\
	Memory\par Optimization & Dead Code Elim, Traceability & Impacts data structures directly \\
	Node Interface & Memory Optimization & Enables \gls{ROS} decoupling \\
	Legacy Code\par Integration & Node Interface & Most architecture-dependent \\
	\bottomrule
\end{xltabular}


Table~\ref{tab:impl_plan} presents the planned order of feature implementation based on their dependencies. Each feature may require elements developed in previous steps, ensuring a coherent and manageable progression. This dependency structure prevents rework by implementing features that provide necessary infrastructure before higher-level functionalities.


\subsection{Core and Metadata Layer}
\label{sec:code_and_meta}

The Core and Metadata Layer (Table~\ref{tab:core_and_meta}) includes foundational features that establish the basic code structure and metadata necessary for subsequent processing. Features such as identifiers and comments form the backbone of the generator, while traceability and reporting provide essential tools for debugging and verification. Their relatively low to medium complexity allows them to be developed early, enabling smooth integration of more advanced features later.

\bgroup
\rowcolors{1}{}{GhostWhite}
\begin{xltabular}{\textwidth}{X X X}
	\caption{Core and Metadata features and their complexity}
	\label{tab:core_and_meta}\\
	\toprule
	\rowcolor{Gainsboro}%
	Feature   & Complexity  & Notes \\
	\midrule
	Identifiers & Low & Nearly complete. \\
	Comments & Very Low & Can parallel identifier completion. \\
	Traceability & Medium & Requires structured tagging throughout the generator. \\
	Report & Low-Medium & Build once traceability is present; not deeply complex. \\
	\bottomrule
\end{xltabular}


This first layer will serve as a base for other features to come since it has fairly straightforward but very useful features.


\subsection{Structural Optimization Layer}
\label{sec:struct_opt_layer}

The Structural Optimization Layer (Table~\ref{tab:struct_opt_layer}) focuses on improving code quality and efficiency. Standard compliance ensures generated code meets coding norms and best practices. Dead code elimination and memory optimization reduce code bloat and resource consumption but require stable metadata and analysis frameworks built in previous layers. These features have higher complexity, reflecting the need for careful analysis and manipulation of generated code.

\bgroup
\rowcolors{1}{}{GhostWhite}
\begin{xltabular}{\textwidth}{X X X}
	\caption{Structural Optimization Layer}
	\label{tab:struct_opt_layer}\\
	\toprule
	\rowcolor{Gainsboro}%
	Feature & Depends On & Notes \\
	\midrule
	Standard Compliance & Report, Traceability & Defining and checking rules\par requires steady effort. \\
	Dead Code Elimination & Traceability, Report & Non-trivial static analysis\par in model-to-code context. \\
	Memory Optimization & Dead Code Elim, Traceability & Impacts core data structure generation.\par Error-prone. \\
	\bottomrule
\end{xltabular}

The structural layer cements what was done before and adds to it with an optimization focused approach.

\subsection{Architecture Layer}
\label{sec:arch_layer}

The Architecture Layer (Table~\ref{tab:arch_layer}) introduces features that abstract and modularize the generated code to support scalability and maintainability. The node interface abstraction separates platform-specific code (\gls{ROS} dependencies) from core logic, facilitating reuse and easier updates. Legacy code integration is the most complex and architecture-dependent feature, involving interfacing with existing external codebases, which requires careful design and planning to avoid integration pitfalls.

\bgroup
\rowcolors{1}{}{GhostWhite}
\begin{xltabular}{\textwidth}{X X X}
\caption{Architecture Layer}
\label{tab:arch_layer}\\
\toprule
\rowcolor{Gainsboro}%
Feature & Depends On & Notes \\
\midrule
Node Interface & Memory Optimization & Architecture-dependent.\par Needs careful planning. \\
Legacy Code Integration & Node Interface & Undefined scope and\par likely the most architecture-sensitive \\
\bottomrule
\end{xltabular}

The last layer of implementation has a higher degree of abstraction and requires a much more project architecture understanding than the previous layers, which is why it is done last and after every other feature layer is implemented. This will allow for the full focus to be on these core features that will have a much higher individual impact on code generation than the previous ones.



