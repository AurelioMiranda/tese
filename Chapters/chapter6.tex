%!TEX root = ../template.tex
%%%%%%%%%%%%%%%%%%%%%%%%%%%%%%%%%%%%%%%%%%%%%%%%%%%%%%%%%%%%%%%%%%%%
%% chapter6.tex
%% NOVA thesis document file
%%
%%%%%%%%%%%%%%%%%%%%%%%%%%%%%%%%%%%%%%%%%%%%%%%%%%%%%%%%%%%%%%%%%%%%

\typeout{NT FILE chapter6.tex}%

\chapter{Software Testing}
\label{cha:test}


\epigraph{ \textit{This chapter documents the software testing done throughout the development of the thesis. It highlights how the specific tests where done, in what context and their results.}}


Intro intro intro?


\section{Feature Testing}
\label{sec:test_feature}

This type of testing is recurrent along the development, it serves to ensure that the features implemented are functioning as intended and that the previous version of the software has retained its functionality when new features were added.

\section{Feature Testing}
\label{sec:test_feature_one}

After the development of the first batch of features, the first testing served to ensure that the previous version of the code generator was still working as intended.





