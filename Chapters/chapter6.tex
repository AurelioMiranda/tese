%!TEX root = ../template.tex
%%%%%%%%%%%%%%%%%%%%%%%%%%%%%%%%%%%%%%%%%%%%%%%%%%%%%%%%%%%%%%%%%%%%
%% chapter6.tex
%% NOVA thesis document file
%%
%%%%%%%%%%%%%%%%%%%%%%%%%%%%%%%%%%%%%%%%%%%%%%%%%%%%%%%%%%%%%%%%%%%%

\typeout{NT FILE chapter6.tex}%

\chapter{Discussion}
\label{cha:discussion}


\epigraph{ \textit{This chapter interprets the results presented in the previous sections in light of the research objectives outlined in Chapter~\ref{cha:introduction}.}}

The objective of this dissertation was to bridge the gap between academic model-driven engineering tools and industrial requirements by introducing a configuration layer to \gls{RAMSES}. The results obtained from the implementation and subsequent testing phases provide significant insights into the validity of the proposed \gls{DSL}.


\section{Interpretation of Technical Results}
\label{sec:technical_results}

The implementation successfully transitioned \gls{RAMSES} from a rigid, hardcoded generator to a configurable platform. The integration tests (Section~\ref{sec:test_feature}) confirmed that the introduction of the configuration layer does not compromise the functional correctness of the generated code.

A critical technical achievement was the decoupling of Business Logic from the \gls{ROS} framework. As observed in the test results, this separation allows developers to reuse C++ logic independently of the middleware, addressing a major request from the industrial partner. Furthermore, the inclusion of Legacy Library Integration solves the "island of automation" problem, where generated code previously failed to interface with existing company ecosystems.

Regarding performance, the analysis of execution times (Section~\ref{sec:exec_times}) indicates that while the configurable workflow introduces a processing overhead, raising the average generation time to approximately 479.9ms compared to the estimated 231.6ms of the legacy generator, this increase is negligible in a human centric workflow. The overhead is primarily driven by the Report and Traceability modules, which provide high-value artifacts that were previously absent. The trade off between a brief delay and the automated production of traceability artifacts and quality reports is highly favorable for safety-critical development contexts.



\section{Usability and Developer Experience}
\label{sec:usability_and_dev_experience}

The usability validation (Section~\ref{sec:test_val}) addressed the concern that introducing a \gls{DSL} might add cognitive complexity to the workflow. The \gls{SUS} scores, averaging [TODO: insert score], suggest that the system falls within the [TODO: insert grade, "Good" or "Excellent"] range of usability.

Qualitative feedback highlighted that the naming convention configuration  significantly improves team cohesion by enforcing coding standards automatically. [TODO: verify The participation of a ROS expert (P5) validated that the configuration options, specifically the thread executor selection and node interface generation, align with real-world industrial expectations.]

However, the study also revealed that users without prior Eclipse experience faced initial navigation challenges. This suggests that while the \gls{DSL} itself is intuitive, its integration into the IDE preferences requires clear documentation or onboarding to minimize the initial learning curve.



\section{Addressing Industrial Constraints}
\label{sec:industrial_constraints}

The solution directly addresses the "Inflexibility of \gls{RAMSES}" identified in the Problem Statement. By externalizing generation policies, the tool now supports:

\begin{itemize} 
	\item \textbf{Compliance: } Through automated naming conventions, file headers and an automatic code reviewer.
	\item \textbf{Traceability: } By mapping \gls{AADL} components to specific lines of code via \gls{JSON} and \gls{HTML} reports.
	\item \textbf{Adaptability: } Through the "overwrite" and "build integration" features, allowing \gls{RAMSES} to fit into incremental build pipelines rather than forcing a clean-build approach.
\end{itemize}

These industry favored features along with the many others implemented bring the current version of \gls{RAMSES} a step closer to industry level tools.



\section{Limitations}
\label{sec:limitations}

As noted in Section~\ref{sec:research_limitations}, the study was limited by the sample size of the usability test and the scope of the supported features. While the \gls{DSL} supports C++, full support for other standards mentioned in the state of the art, such as \gls{AUTOSAR}, remains a future goal. Additionally, the dependency on third-party libraries for compliance (CppCheck) means the tool's effectiveness is partially bound by external tool capabilities in the realm of code compliance.







